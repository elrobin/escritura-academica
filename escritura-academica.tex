% Options for packages loaded elsewhere
\PassOptionsToPackage{unicode}{hyperref}
\PassOptionsToPackage{hyphens}{url}
%
\documentclass[
]{book}
\title{Cómo redactar un artículo científico en Ciencias Sociales}
\author{Nicolás Robinson-Garcia}
\date{2022-03-21}

\usepackage{amsmath,amssymb}
\usepackage{lmodern}
\usepackage{iftex}
\ifPDFTeX
  \usepackage[T1]{fontenc}
  \usepackage[utf8]{inputenc}
  \usepackage{textcomp} % provide euro and other symbols
\else % if luatex or xetex
  \usepackage{unicode-math}
  \defaultfontfeatures{Scale=MatchLowercase}
  \defaultfontfeatures[\rmfamily]{Ligatures=TeX,Scale=1}
\fi
% Use upquote if available, for straight quotes in verbatim environments
\IfFileExists{upquote.sty}{\usepackage{upquote}}{}
\IfFileExists{microtype.sty}{% use microtype if available
  \usepackage[]{microtype}
  \UseMicrotypeSet[protrusion]{basicmath} % disable protrusion for tt fonts
}{}
\makeatletter
\@ifundefined{KOMAClassName}{% if non-KOMA class
  \IfFileExists{parskip.sty}{%
    \usepackage{parskip}
  }{% else
    \setlength{\parindent}{0pt}
    \setlength{\parskip}{6pt plus 2pt minus 1pt}}
}{% if KOMA class
  \KOMAoptions{parskip=half}}
\makeatother
\usepackage{xcolor}
\IfFileExists{xurl.sty}{\usepackage{xurl}}{} % add URL line breaks if available
\IfFileExists{bookmark.sty}{\usepackage{bookmark}}{\usepackage{hyperref}}
\hypersetup{
  pdftitle={Cómo redactar un artículo científico en Ciencias Sociales},
  pdfauthor={Nicolás Robinson-Garcia},
  hidelinks,
  pdfcreator={LaTeX via pandoc}}
\urlstyle{same} % disable monospaced font for URLs
\usepackage{longtable,booktabs,array}
\usepackage{calc} % for calculating minipage widths
% Correct order of tables after \paragraph or \subparagraph
\usepackage{etoolbox}
\makeatletter
\patchcmd\longtable{\par}{\if@noskipsec\mbox{}\fi\par}{}{}
\makeatother
% Allow footnotes in longtable head/foot
\IfFileExists{footnotehyper.sty}{\usepackage{footnotehyper}}{\usepackage{footnote}}
\makesavenoteenv{longtable}
\usepackage{graphicx}
\makeatletter
\def\maxwidth{\ifdim\Gin@nat@width>\linewidth\linewidth\else\Gin@nat@width\fi}
\def\maxheight{\ifdim\Gin@nat@height>\textheight\textheight\else\Gin@nat@height\fi}
\makeatother
% Scale images if necessary, so that they will not overflow the page
% margins by default, and it is still possible to overwrite the defaults
% using explicit options in \includegraphics[width, height, ...]{}
\setkeys{Gin}{width=\maxwidth,height=\maxheight,keepaspectratio}
% Set default figure placement to htbp
\makeatletter
\def\fps@figure{htbp}
\makeatother
\setlength{\emergencystretch}{3em} % prevent overfull lines
\providecommand{\tightlist}{%
  \setlength{\itemsep}{0pt}\setlength{\parskip}{0pt}}
\setcounter{secnumdepth}{5}
\usepackage{booktabs}
\usepackage{amsthm}
\makeatletter
\def\thm@space@setup{%
  \thm@preskip=8pt plus 2pt minus 4pt
  \thm@postskip=\thm@preskip
}
\makeatother
\ifLuaTeX
  \usepackage{selnolig}  % disable illegal ligatures
\fi
\usepackage[]{natbib}
\bibliographystyle{apalike}

\begin{document}
\maketitle

{
\setcounter{tocdepth}{1}
\tableofcontents
}
\hypertarget{bienvenidoa}{%
\chapter*{Bienvenido/a}\label{bienvenidoa}}
\addcontentsline{toc}{chapter}{Bienvenido/a}

\includegraphics{images/academic-writing.jpg}

\textbf{Curso:} Cómo escribir y publicar artículos científicos en Ciencias Sociales y Humanidades.

\textbf{Organiza:} Escuela Internacional de Posgrado de la Universidad de Granada

\textbf{Imparten:} \href{https://www.ugr.es/en/staff/evaristo-jimenez-contreras}{Evaristo Jiménez-Contreras}, \href{https://sites.google.com/go.ugr.es/torressalinas}{Daniel Torres-Salinas} y \href{https://nrobinsongarcia.com}{Nicolás Robinson-García}

Esta página web contiene el material impartido por Nicolás Robinson-García para la sesión sobre redacción científica. En esta sesión trataremos diferentes cuestiones prácticas sobre la preparación del manuscrito académico, edición y estilo.

\hypertarget{intro}{%
\chapter{Introducción}\label{intro}}

Filosofía general: ¿por qué escribir bien y por qué publicar?
- El objetivo de la investigación
- Mini recorrido histórico del sistema de publicación científica

\hypertarget{las-fases-de-la-carrera-acaduxe9mica}{%
\section{Las fases de la carrera académica}\label{las-fases-de-la-carrera-acaduxe9mica}}

\begin{itemize}
\tightlist
\item
  Las tres comunidades de Gläser y Laudel
\end{itemize}

\hypertarget{el-cientuxedfico-como-escritor-obrero}{%
\section{El científico como escritor obrero}\label{el-cientuxedfico-como-escritor-obrero}}

Las lecciones de Stephen King:
- Nadie te considerará para un Nobel de literatura, pero serás el más leído

\hypertarget{resistencia-al-rechazo}{%
\section{Resistencia al rechazo}\label{resistencia-al-rechazo}}

\begin{quote}
`this is good. Not for us, but good. You have talent. Submit again'
\end{quote}

\begin{itemize}
\tightlist
\item
  Aprende del rechazo y tómatelo con deportividad
\item
  Conócete a tí mismo y encuentra tu voz
\end{itemize}

\hypertarget{escribe-para-tuxed-revisa-para-los-demuxe1s}{%
\section{Escribe para tí, revisa para los demás}\label{escribe-para-tuxed-revisa-para-los-demuxe1s}}

\begin{quote}
when you write a story, you're telling yourself the story {[}\ldots{]} When you rewrite, your main job is taking out all the things that are not the story
\end{quote}

\hypertarget{el-artuxedculo-cientuxedfico}{%
\chapter{El artículo científico}\label{el-artuxedculo-cientuxedfico}}

\hypertarget{quuxe9-dice-la-apa}{%
\section{¿Qué dice la APA?}\label{quuxe9-dice-la-apa}}

\hypertarget{tipos-de-trabajos-cientuxedficos}{%
\section{Tipos de trabajos científicos}\label{tipos-de-trabajos-cientuxedficos}}

\hypertarget{la-elecciuxf3n-del-tema-de-investigaciuxf3n}{%
\chapter{La elección del tema de investigación}\label{la-elecciuxf3n-del-tema-de-investigaciuxf3n}}

\hypertarget{anatomuxeda-de-una-idea}{%
\section{Anatomía de una idea}\label{anatomuxeda-de-una-idea}}

\begin{itemize}
\tightlist
\item
  originalidad
\item
  inédito
\item
  novedad
\item
  nada es de nadie hasta que se publica (prioridad científica)
\end{itemize}

\hypertarget{herramientas-de-buxfasqueda-bibliogruxe1fica}{%
\section{Herramientas de búsqueda bibliográfica}\label{herramientas-de-buxfasqueda-bibliogruxe1fica}}

\begin{itemize}
\tightlist
\item
  La serendipia y cómo alimentarla
\item
  El papel de la revisión bibliográfica
\item
  Herramientas de búsqueda
\end{itemize}

\hypertarget{la-autoruxeda-cientuxedfica}{%
\chapter{La autoría científica}\label{la-autoruxeda-cientuxedfica}}

\hypertarget{quuxe9-es-un-autor}{%
\section{¿Qué es un autor?}\label{quuxe9-es-un-autor}}

\begin{itemize}
\tightlist
\item
  Definición según la ICMJE
\item
  El trabajo colaborativo
\item
  Los autores fantasma
\item
  Los autores de cortesía
\end{itemize}

\hypertarget{las-contribuciones}{%
\section{Las contribuciones}\label{las-contribuciones}}

\begin{itemize}
\tightlist
\item
  Nuevo paradigma de autoría
\item
  Tipos de autores según su contribución
\item
  Los riesgos de la hiper-especialización
\item
  Taxonomía CREDIT
\end{itemize}

\hypertarget{organizaciuxf3n-y-estructura-del-documento}{%
\chapter{Organización y estructura del documento}\label{organizaciuxf3n-y-estructura-del-documento}}

\hypertarget{introducciuxf3n}{%
\section{Introducción}\label{introducciuxf3n}}

\hypertarget{material-y-muxe9todos}{%
\section{Material y métodos}\label{material-y-muxe9todos}}

\hypertarget{resultados}{%
\section{Resultados}\label{resultados}}

\hypertarget{discusiuxf3n}{%
\section{Discusión}\label{discusiuxf3n}}

\hypertarget{secciones-adicionales}{%
\section{Secciones adicionales}\label{secciones-adicionales}}

\hypertarget{antecedentesrevisiuxf3n-bibliogruxe1fica}{%
\subsection{Antecedentes/Revisión bibliográfica}\label{antecedentesrevisiuxf3n-bibliogruxe1fica}}

\hypertarget{marco-conceptualhipuxf3tesis}{%
\subsection{Marco conceptual/Hipótesis}\label{marco-conceptualhipuxf3tesis}}

\hypertarget{objetivos}{%
\subsection{Objetivos}\label{objetivos}}

\hypertarget{conclusiones}{%
\subsection{Conclusiones}\label{conclusiones}}

\hypertarget{cuestiones-de-estilo}{%
\chapter{Cuestiones de estilo}\label{cuestiones-de-estilo}}

\hypertarget{la-historia}{%
\section{La historia}\label{la-historia}}

\hypertarget{estilo-y-citaciuxf3n}{%
\section{Estilo y citación}\label{estilo-y-citaciuxf3n}}

\begin{itemize}
\tightlist
\item
  Por qué citar
\item
  Cuándo citar
\item
  Qué citar
\item
  Cómo citar
\end{itemize}

\hypertarget{conoce-a-tu-puxfablico}{%
\section{Conoce a tu público}\label{conoce-a-tu-puxfablico}}

\hypertarget{tablas-y-figuras}{%
\chapter{Tablas y figuras}\label{tablas-y-figuras}}

\hypertarget{duxf3nde-y-cuuxe1ntas-figuras-incluir}{%
\section{¿Dónde y cuántas figuras incluir?}\label{duxf3nde-y-cuuxe1ntas-figuras-incluir}}

Utiliza las figuras como un recurso discursivo para:
Sintetizar y resumir procesos. Ejemplo
Facilitar la interpretación de resultados. Ejemplo
Mostrar ejemplos a tipo ilustrativo. Ejemplo

\hypertarget{caracteruxedsticas-buxe1sicas-de-una-tablafigura}{%
\section{Características básicas de una tabla/figura}\label{caracteruxedsticas-buxe1sicas-de-una-tablafigura}}

La información de tablas y figuras no debe ser redundante.
Las figuras deben ser explanatorias en interpretables por sí solas
Cuida aspectos que puedan llevar a malas interpretaciones (e.g., escala de los ejes)
Invierte tiempo en preparar figuras bonitas.
Figuras con excel: Ejemplo feo, ejemplo bonito.
Figuras con R: Ejemplo feo, ejemplo bonito

\hypertarget{software-para-crear-figuras}{%
\chapter{Software para crear figuras}\label{software-para-crear-figuras}}

\hypertarget{ediciuxf3n-y-preparaciuxf3n-del-manuscrito}{%
\chapter{Edición y preparación del manuscrito}\label{ediciuxf3n-y-preparaciuxf3n-del-manuscrito}}

\hypertarget{los-metadatos}{%
\section{Los metadatos}\label{los-metadatos}}

\begin{itemize}
\tightlist
\item
  Portada
\item
  Título
\item
  Resumen
\item
  Palabras Clave
\end{itemize}

\hypertarget{formatos-de-revista-y-plantillas}{%
\section{Formatos de revista y plantillas}\label{formatos-de-revista-y-plantillas}}

\begin{itemize}
\tightlist
\item
  Referencias bibliográficas
\item
  Overleaf y rticles
\end{itemize}

\hypertarget{agradecimientos-y-financiaciuxf3n}{%
\chapter{Agradecimientos y financiación}\label{agradecimientos-y-financiaciuxf3n}}

\begin{itemize}
\tightlist
\item
  Pide a tus colegas que revisen tus manuscritos o secciones específicas y luego agradéceselo
\item
  Indica si has recibido financiación y por parte de qué entidad
\item
  Si versiones preliminares o algunos resultados fueron presentados en conferencias o seminarios de investigación, inclúyelo también
\item
  Sé agradecido con los revisores (si se lo merecen)
\item
  También sé agradecido con compañeros que te hayan ayudado en la confección del trabajo pero no sean autores
\end{itemize}

  \bibliography{book.bib}

\end{document}
